\documentclass{article}
\usepackage[authordate, backend=biber, annotation]{biblatex-chicago}
\usepackage[margin=1in]{geometry}
\usepackage[]{epigraph}
\addbibresource{paper.bib}

\usepackage [english]{babel}
\usepackage [autostyle, english = american]{csquotes}
\MakeOuterQuote{"}


\title{An Oral History Project: Coke Studio and the Revitalization of Eastern Music}
\date{April 2022}
\author{Sarim Aleem \\ Word Count: 2135}

\setlength\epigraphwidth{.8\textwidth}
\setlength\epigraphrule{0pt}

  \begin{document}
  \maketitle

  
  \section{Introduction}

  \epigraph{\itshape One day, I suddenly discovered eastern music. It was a very
  strange day for me, you know when something inside just goes click.}{--- Rohail Hyatt}

  
  Like Rohail Hyatt, most Pakistani's aren't aware of the one thousand year
  heritage of music and culture that in the diaspora. Most would ignore the rich
  classical music scene for being boring or inaccessible, and would most likely
  rather opt for bollywood style dance videos - a dime a dozen in the country. 

  But as Hyatt, the leader of a reignition of a rich and vibrant traditional
  culture would discover, there is an incredible philosophical and poetic depth
  to traditional music.  But little did he expect that the reignition of that
  lost culture would not come from the local film or entertainment industry.
  Rather, it would come from a well known Western soft drink company, initially
  trying to outcompete its famed rival through a small marketing stint: Coca
  Cola. 

  
  The introduction of Coke Studio in Pakistan has led to a revitalization of
  classical music by reintroducing Sufi culture, putting a diverse catalog of
  artists in music, and having a nontrivial amount of linguistic diversity in
  songs.  Nevertheless, the production studio has it's flaws characteristic of a
  large capitalistic coorporation, and oftentimes commoditizes classic culture
  and neglects it's heritage. 

  \section{Background}

  It began as a marketing project in Pakistan. Pepsi, Coca Cola's main rival
  over 70\% of the soft drink market share. Pepsi's success was largely due to
  its association with Cricket and Music in the mind of most Pakistanis: "Be it
  through sponsorships, endorsements with big cricket stars, investing in new
  emerging cricketing and music talent, Pepsi was successful in becoming a
  global brand with local connection" \autocite{Chibmedium}. With growth 
  potential low in more developed countries, Coca Cola saw an opportunity to 
  grow its product in developing countries. 

  Still though, most Pakistani were not terribly interested in traditional folk 
  music given the wave of westernization with the internet. But Rohail Hyatt had a 
  different idea. Strike at Pakistani cultural pride in the face of globalization, but 
  give audiences a taste of modernity within the folk music to draw them in. Thus, 
  \textit{Fusion} was born, a term coined by Hyatt to refer to the mix between modern pop music
  and traditional folk \autocite{dhanwani2014coke}.

  And then, Coke studio was born; blending East and West when it came to musical
  instruments, culture, and musical character. This blend would not only
  revitalize traditional culture but for better of worse, innovate on it.

  \section{A Reintroduction of Sufi Culture}

  There are three main branches in the religion of Islam. Sunni - the normative
  version of Islam practiced by the majority of the world. Shi'a - a smaller but
  not insignificant sect with minor differences in religious interpretation. And
  finally, there is Sufi Islam. Although technically Sufi Islam is not
  considered by some to be a separate branch of Islam, many of their practices
  are not found in normative Islam and thus, for the purposes of this paper they
  will be considered a different "branch" or "sect" \autocite{sufibackground}. 

  Sufi Islam is commonly defined "Islamic Mysticism" and tends to emphasize the 
  spiritual aspects of the religion more so than the legalistic, or worldy
  elemnts. For most orientalists, the religion is thus seen as the more 
  moderate, "peaceful" interpretation of islam. This of
  course incorrectly implies that traditional mainstream islam is oppressive,
  but nevertheless the perception of Westerners ironically have a large
  influence on what local Pakistanis wish to express their culture \autocite{beg2020fizzy}. As a
  consequence, embracing the Sufi genre of music - the Qawwali - is a nice
  middle ground in appeasing to a moderate image of Islam while simultaneously 
  appealing to the more religiously inclined in the nation who are skeptical of
  the value of music in society \autocite{williams2019soundtrack}.

  The Qawwali is a musical manifestation of Sufism. It's a piece of religious music 
  that tends to be 10-20 minutes long, but can range up to an hour and is typically 
  based on religious poetry. The piece starts out slowly, with a couple words chanted
  in the begining, but quickly accelerates to a climax. 

  Coke studio's rendition of the 800 year old famous poem "Chaap Tilak" is a
  quintissential example of such Qawwali \autocite{Chaaptilaak}. The song starts
  slow, and the singers are chanting one word over the course of 10-15 seconds.
  The slow introduction serves the purpose to "prime" those who don't understand
  the language, as the music can be appreciated by anyone. The song slowly
  accelerates to a climax so intense the singers start sweating profusely. 

  As with most Sufi poetry, the lyrical content of Chaap Tilak is very divorced from 
  political reality, and is far more focused on abstract and "higher" concepts:

  \begin{quote}
    
    The Game of Love
    The Game of Love
    If I play with my dear one

    If I win, My sweetheart is Mine
    If I lose, I'm still with my Dear one

    \dots

    The night of the Prophet's celebration has dawned
    The night has dawned
    The night has dawned

    \dots

    With one glance your enchanting eyes

  \end{quote}

  Love and devotion to God featured heavily in the song, and much of Sufi
  Qawwali is described as eschewing wordly and material life - hardly a
  controversial message. For this reason, Qawwali's are a popular genre to
  market to worldwide audiences

  What's important to note here is not only the song itself, but its reception. 
  The video itself has almost \textit{fifty million views} on Youtube, extremely
  high for a rendition of a classical poem. Furthermore, the comment section
  seems to affirm the sublime superiority of classical culture over modernity.
  One user writes: "Maturity is when you understand classical music is apart
  from all genre." Another comments points to the fact that the music trancends political
  differences in the subcontinent. One user writes: "I'm a Pakistani. but I can
  see here most of the comments are from Indians It shows Indian are really
  music lover and they know how to appreciate talent." 

  That's not to say that everyone is happy with the reintroduction of Sufism in modern culture. 
  For many, the reintroduction of Sufi culture seems commodotized. As Amin writes: 
  
  \begin{quote}
    What happened to qawwali in this process? [After it's greater presence in
    90's] It was unmoored from its deeper spiritual significance and repackaged
    as a cultural item, all so that it could be more easily commodified for a
    global audience. \autocite{sellingsufism}
  \end{quote}

  For many, Coke Studio adopting the Qawwali was not seen as a victory, but
  rather a capitalist society appropriating a millenia old tradition for profit.
  In fact, although the Qawwali comes from the Islamic tradition, "Qawwali is
  rarely marketed as Islamic music because to do so would risk turning away
  consumers, as Islam purportedly does not share the same peaceful values as
  Sufism" \autocite{sellingsufism}. So in essence, a company like Coke Studio 
  takes the Qawwali, and erases it's cultural heritage when convenient to make 
  profits - Hardly an ode to authenticity. 

  Part of the reason that the genre of Sufi music became so popular was that it
  largely touched on the spiritual aspects of religion. Another part is that the
  poem predates the beginning of political differences in the Indian
  subcontinent. And yet another is the fact that The video is freely available
  on youtube, arguably one of the largest if not the largest video streaming
  site in the world. Combined with Coca Cola's large marketing budget, it's not then 
  surprising that Qawwalis became the hits they did in Coke Studio. 

  \section{Women and Young Performers in Coke Studio}

  Although Coke Studio has bought Sufi music into the mainstream, there are aspects
  of it that differ greatly from its traditional performance. 

  On a reexamination of the video above, one may notice that Abida Parveen is a woman. 
  While it's not particularly unusual for a woman to sing in a Western context, it's 
  certainly the case that when it comes to Islamic religious veneration, women are 
  traditionally not "leaders." For this reason, men traditionally led Qawwali's.

  Take for instance, Naz Warsi, who is a trained female musician in the Sufi tradition. 

  \begin{quote}

    Warsi told the reporter that although her desire - similar to other female
    performers - is to perform qawwali, they are not granted permission.
    Although Warsi has obtained training in Sufi classical singing and has
    earned her name in the genre, it is twice the struggle for her to take her
    place in the world of qawwali. She is not only a woman but a woman who does
    not possess the “proper” musical lineage (BBC Radio Urdu, n.d.).  It said
    that only men should be permitted to sing qawwali but that men who have
    inherited the profession should be the only ones to perform. The role of
    females is actively discouraged within sacred dargah culture. Only male
    heirs are to succeed their fathers, making paternal lineage the trait that
    allows a singer to become a performer in a Sufi dargah
    \autocite{beg2020fizzy}.

  \end{quote}

  In order to appeal the a large segment of the modern population, Coke Studio
  has taken to the approach of of intermixing traditional Sufi music with modern
  styles. One way it does this is by not only adding women, but also young
  people not traditionally thought to have sung the songs. 

  Take for example, arguably the most popular Coke Studio hit: "Afreen."

  Afreen is a quintissential example of Coke studio subverting expectations when
  it came to artist selection. The song was originally sung by the dubbed "King
  of Qawwali" Nusrat Fateh Ali Khan. The cover on the other hand was not only
  sung by his nephew, the esteemed Rahat Fateh Ali Khan, but also by Momena
  Mostehsan: a young woman raised in the United States \autocite{afreen}. The combining of
  cultures brings Sufi culture into the modern era, and thus there's a sense of
  innovation in the music. Furthermore, not only is the artist selection crucial,
  but also their appearance. 

\begin{quote}
  what garnered more attention sometimes were the looks of the artist, than the
  artist himself/ herself. One would often find the folk musician accompanied by a
  female singer in jeans, cropped leather jacket, painted nails, which was indeed
  a counter narrative to the burkha wearing Islam women \autocite{Chibmedium}.
\end{quote}

The inclusion of women and nontraditional artists is important because it
signals a change in how classical music is performed. The music is stripped down
to it's bare minimum and reconstructed. And, for the most part it has succeeded.
Songs with these new and modern artists have hundreds of millions of views on
youtube, and have captivated the imagination of the diaspora. 

\section{The language of Coke Studio}

Although Coke Studio has bought an incredible amount of innovation of into the
music scene, there are reminders that at the end of the day the studio is just a
marketing stint for a multi-national coorporation. As such, there can be serious
oversights when it comes to what's performed and what isn't. One such oversight is 
clearly present when it comes to the issue of language. 

Unlike the United States, which is mostly homogenous in terms of linguistic
diversity, the Hindustani region has an incredible amount of languages and
subcultures splintered in.  There are over 30 languages alone with over 1
million speakers in the subcontinent, and about 122 languages with more than
10,000 speakers.

It's then quite unfortunate that Coke Studio is lacking in the language
department, and fails to cover the linguistic diversity of the subcontinent. 

Rodrigo Chocano, a Peruvian enthnomusicologist writes: 

\begin{quote}
  Between the two [Urdu and Punjabi], they account for way over two thirds of the whole pie.

  Fifteen languages split the remaining spots. In comparison to Punjabi, the other
  three major regional languages pale: while Pashto, Balochi and Sindhi together
  account for over one third of native speakers in Pakistan, they only amount to
  four percent of the linguistic presence in Coke Studio Pakistan \autocite{soundscape}.
\end{quote}


That's not to say that Coke Studio has no linguistic diversity. The producers
are humans after all and do make attempts to put in some sort of linguistic
diversity.  For example the following languages and scripts have been learnt and
used by the producers: "Seraiki, Sindhi, Pashto, Punjabi, Brahui, Balochi, Braj
Bhasha, Marwari, Bengali, Urdu, Persian, Arabic and Turkish" \autocite{herald}.
That is still a great deal more than is present in bollywood media. 

Still, It's hard to forget that at the end of the day, Coca Cola is a business. It's
simply more convenient to appeal to the "prestige" languages that are used in
Poetry (Persian) or Pop Culture (Urdu, Punjabi) than go out of the way to produce 
music in other languages \autocite{ethnographicInterview}. 

Although this may look harmless at the outset, if Coke Studio is viewed, as
perhaps many do, as being an musical representation of the sould of the nation,
then many niche but equally important subcultures can be overlooked, and the
perception of culture is then formed by a media company that is, for lack of
better terms, "in it for the money." Unfortunately, money isn't concerned with
diversity or representation if said diversity doesn't help the bottom line.

But still, the issue becomes worse when poetry from other languages is sung
incorrectly: 

\begin{quote}
 It's not just that certain nuances are getting lost in what these artists are
 singing. They are actually singing some completely meaningless lines and you
 can imagine what that means for the transmission of meaning for classical
 poetry that has been traditionally passed on from one generation to another \autocite{herald}.
\end{quote}

We tend to see unfortunately that there is a sense of neglect. One megahit that 
coke studio produced is a song by the name "Taj  Da Re Haram"
\autocite{taajdaar}. The song is the most popular song on the youtube channel by
view count (over 374 million views as of this writing).  However, out of
carelessness many of the lyrics are wrong to the point of being nonsensical
\autocite{herald}.

In some part, this failing can be attributed to the fact that education in
traditional languages like Urdu have lessened in recent years, as English is a
far more useful language from a capitalistic standpoint. Popular artists don't 
fact check their songs for accuracy all the time. This is reminiscent of the 
previous discussion on the commodotization of Qawwali's. After all, if a great 
deal of the population is unable to fact check the songs themselves, then it
doesn't hurt the bottomline too much to overlook careful scrutiny that ensures
the classic renditions are authentic. 

Arguably the largest music producer in the country is doing a disservice to 
classic poetry and culture. There is an element of neglect in the rendition of
popular pieces of classic poetry, perhaps hinting at the fact the the priorities of 
Coke Studio are slightly superficial. 

\section{Conclusion}

Although the Coke Studio discussed and the one popularly known is the Pakistani
one, Coca Cola has in fact attempted to establish studios in other countries,
including India. However, the Indian Coke studio never reached the fame the
Pakistani did, since it couldn't escape it's consumerist, capitalist, Bollywood
roots. But, the success of Pakistani Coke Studio points to the fact that in the
face of the globalization and westernization of Pakistan, there is a sense of
pride in sublime of classic poetry and music, however imperfect the modern
rendition of that may be. 

Coke Studios innovation in the field has been nothing short of incredible.
Classic Poetry from almost a millenia ago are being popularized and being
listened to hundreds of millions of times.  However, in the usage of classical
sources, it is imperative to stay faithful to them, which Coke Studio has failed
to do. After all, for many, Coke Studio has become more than a marketing stint; 
it has trancended to Pakistan's oral history project. 

\printbibliography
\end{document}

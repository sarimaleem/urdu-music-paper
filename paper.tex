\documentclass{article}
\usepackage[authordate, backend=biber, annotation]{biblatex-chicago}
\usepackage[margin=1in]{geometry}
\usepackage[]{epigraph}
\addbibresource{paper.bib}

\usepackage [english]{babel}
\usepackage [autostyle, english = american]{csquotes}
\MakeOuterQuote{"}


\title{An Oral History Project: Coke Studio and the Revitalization of Eastern Music}
\date{April 2022}
\author{Sarim Aleem}

\setlength\epigraphwidth{.8\textwidth}
\setlength\epigraphrule{0pt}

  \begin{document}
  \maketitle

  
  \section{Introduction}

  \epigraph{\itshape One day, I suddenly discovered eastern music. It was a very
  strange day for me, you know when something inside just goes click.}{--- Rohail Hyatt}

  
  Like Rohail Hyatt, most Pakistani's aren't aware of the one thousand year
  heritage of music and culture that in the diaspora. Most would ignore the rich
  classical music scene for being boring or inaccessible, and would most likely
  rather opt for bollywood style dance videos, a dime a dozen in the country. 

  But as Hyatt, the leader of a reignition of a rich and vibrant traditional
  culture would discover, there is an incredible philosophical and poetic depth
  to traditional music.  But little did he expect that the reignition of that
  lost culture would not come from the local film or entertainment industry.
  Rather, it would come from a well known Western soft drink company, initially
  trying to outcompete its famed rival through a small marketing stint: Coca
  Cola. And, he would spearhead it. 

  The introduction of Coke Studio in Pakistan has led to the revitalization of
  traditional culture by putting a reintroducing endangered languages, putting a
  spotlight on traditional Eastern instruments, and bringing traditional Sufi
  culture into the modern day. 

  \section{Background}

  It began as a marketing project in Pakistan. Pepsi, Coca Cola's main rival
  over 70\% of the soft drink market share. Pepsi's success was largely due to
  its association with Cricket and Music in the mind of most Pakistanis: "Be it
  through sponsorships, endorsements with big cricket stars, investing in new
  emerging cricketing and music talent, Pepsi was successful in becoming a
  global brand with local connection" \autocite{Chibmedium}. With growth 
  potential low in more developed countries, Coca Cola saw an opportunity to 
  grow its product in developing countries. 

  Still though, most Pakistani were not terribly interested in traditional folk 
  music given the wave of westernization with the internet. But Rohail Hyatt had a 
  different idea. Strike at Pakistani cultural pride in the face of globalization, but 
  give audiences a taste of modernity within the folk music to draw them in. Thus, 
  \textit{Fusion} was born, a term coined by Hyatt to refer to the mix between modern pop music
  and traditional folk \autocite{dhanwani2014coke}.

  Thus, Coke studio was born; blending East and West when it came to musical instruments, culture,
  and musical character. This blend would not only revitalize traditional culture but innovate on it. 

  \nocite{*}
  \printbibliography
\end{document}
